\documentclass[11pt]{article}

\usepackage[parfill]{parskip}
\usepackage{hyperref}

\setlength{\topmargin}{-.5in}
\setlength{\textheight}{9in}
\setlength{\oddsidemargin}{.125in}
\setlength{\textwidth}{6.25in}

\newcommand {\cmd}[1] {\begin{quote}#1\end{quote}}

\begin{document}
\title{Open Genomics Engine - Framework Documentation}
\author{Lee Baker and David Mittelman\\
Virginia Bioinformatics Institute \\
leecb@vt.edu, david.mittelman@vt.edu}
\maketitle

\setcounter{tocdepth}{2}
\tableofcontents

\section {Introduction}
The Open Genomics Engine (OpenGE) is an open software framework for manipulating and interpreting high-throughput sequencing data. This documentation describes core functions that are integrated into the official distribution of the framework. OpenGE is available as binaries for Linux and MacOS or as source. The source distribution contains further instructions for compiling and installing the software. OpenGE now includes Debian and RedHat packages for easier installation.

\section {Using OpenGE}
OpenGE consists of a collection of operations that can be invoked on the command line using the following syntax:

\cmd{openge \textit{command [options]}}

where \textit{command} is one of the commands listed in the following table, and \textit{options} are various flags and parameters that may be supplied depending on the command:

\begin{center}
\begin{tabular}{lp{3.5in}}
\hline
Command&Description\\ \hline
compare&Compare reads in two alignments and reports changes\\
count&Count reads in an alignment\\
coverage&Compute coverage statistics for an alignment\\
dedup&Mark (or remove) duplicate reads\\
execute&Run a workflow script\\
help&View detailed help for any OpenGE command\\
history&Report previous operations applied to an alignment\\
mergesort&Merge and sort multiple inputs\\
localrealign&Realign reads around indels\\
repeatseq&Report microsatellite genotypes from an alignment\\
stats&Report statistics for one or more alignments\\
version&Report the OpenGE version number\\
view&View a filtered subset of an input dataset\\
\end{tabular}
\end{center}

\subsection {General}
\label{general_options}
The following flags are common for all OpenGE commands. Additional flags and parameters may be available for individual commands.
\begin{center}
\begin{tabular}{llp{3.5in}}
\hline
Flag&Long flag&Description\\ \hline
-i \textit{filename}&{-}{-}in \textit{filename}&Input filename. Input files can also be specified without this flag. Defaults to stein if no input is specified.\\
-o \textit{filename}&{-}{-}out \textit{filename}&Output filename. Defaults to stdout if omitted. \\
-v&{-}{-}verbose&Display additional information to stderr during operation (e.g. progress indicators and other informational messages). Optional.\\
-c \textit{level}&{-}{-}compression \textit{level}&Compression level- defaults to 6. Valid levels are 0-9, and correspond to zlib's deflate compression levels. Optional.\\
-T&{-}{-}tmpdir&Specify a directory to store temporary files. (default /tmp) Optional.\\
-t&{-}{-}threads&Set the number of threads to be used for multithreaded operations. Optional.\\
-d&{-}{-}nothreads&Disable multithreading. Optional.\\
&{-}{-}nosplit&Disable splitting by chromosome (see below). Optional.\\
-F \textit{format}&{-}{-}format \textit{format}&Select file format. Optional.\\
&{-}{-}nopg \textit{format}&Do not append an \@PG record to any generated BAM or SAM files. \\
\end{tabular}
\end{center}

When running some commands, OpenGE may process chromosomes in separate threads in order to increase speed. Whether or not this occurs depends on the number of cores available on your machine. This increases the amount of memory consumed- you can disable this with the {-}{-}nosplit option.

When input or output files are required, stdin or stdout may be used by simply omitting \textit{filename}. For example, the following command:

\cmd{openge mergesort}

with no input specified, will read a file from stdin, operate on the file, and write the results to stdout. This enables commands to be chained together. For instance:

\cmd{openge dedup in.sam $|$ openge count}

will count the number of lines in a SAM file. Console redirection can be used as well. The following two commands are equivalent in a bash shell:

\cmd{openge mergesort a.bam -o c.bam}
\cmd{openge mergesort $<$ a.bam $>$ c.bam}

\subsection {File formats}

OpenGE supports both BAM and SAM formats as inputs for all commands. By default, results are written in BAM format; experimental SAM and FASTQ support is also present. To write in SAM format, specify a filename with a .sam or .SAM extension; for FASTQ format, use a .fastq or .FASTQ extension. Alternatively, if you can specify the -F \textit{format} parameter if passing output to stdout. For example, to pipe openge's output to another program in FASTQ format, use the following:

\cmd{openge \textit{command} in.bam -F fastq | \textit{some\_command}}

Most commands support multiple files as inputs; these files should have identical headers; corrupted data may result from processing multiple BAM or SAM files with different headers. A warning will be displayed if the headers in your input files are not identical.

Conversion between formats can be performed with the view command. For instance:
\cmd{openge view in.bam out.sam}
converts the file in.bam to SAM format.

\cmd{openge view in.bam out.fastq}
converts the file in.bam to FASTQ format.

\subsubsection {Selecting the output format}
OpenGE tries to automatically select the correct output format by guessing from the filename. Using extensions of .bam, .sam and .fastq (case insensitive) for the output filename will result in the appropriate file format being selected. Alternatively, you can specify a format with the {-}{-}format parameter. If there is neither a valid extension or a format specified on the command line, the output defaults to BAM (or SAM if using the view command and writing to stdout).

To summarize, the following methods are used to select file type (in order of preference):
\begin{itemize}
\item If a file format is specified on the command line with {-}{-}format, the specified format is used
\item If the file has an extension of .bam, .sam, or .fastq (case insensitive), that format is used
\item If the command is view, and data is being written to stdout, SAM format is used for output
\item Finally, BAM format is used for anything else
\end{itemize}

\subsection{FASTA reference format}
Some commands require a FASTA reference file; if necessary, it may be provided with \textit{-R filename}. OpenGE will use a FASTA index if present; if not, an index is automatically created.

\section {Commands}

\subsection {compare}
The compare command compares the reads present in two or more different files to detect when a file has reads that have changed, been added, or been removed. If more than two files are provided, the differences between the first and each subsequent file are also printed.

\begin{center}
\begin{tabular}{llp{3.5in}}
\hline
Flag&Long flag&Description\\ \hline
-p&{-}{-}print&Output coverage data base-by-base to a TSV file.\\
-r  \textit{region}&{-}{-}region \textit{region}&Regions to compare.\\
\end{tabular}
\end{center}

When no region is specified, the whole file is compared.

Multiple regions may be provided- including overlapping regions. The region string format is described in the "Region string format" section below.

For example,
\cmd{openge compare a.bam b.bam}
prints the number of differences between the reads in a.bam and b.bam.
\cmd{openge compare --print b.bam c.bam d.bam}
prints between b.bam and c.bam, and then b.bam and d.bam, inluding printing out the location of each change. You can also limit the comparison to a specified region:
\cmd{openge compare --print b.bam c.bam d.bam -r YHet:1..200 -r U:100..400}

\subsection {count}
The count command returns the number of reads contained in one or more files. No parameters (other than input files) are required.

Example:
\cmd{openge count a.bam}
\cmd{openge count b.bam c.bam d.bam}

\subsection {coverage}
The coverage command reports the read depth as a function of reference genome location. 

Optionally, detailed information on the coverage of every base of the reference genome can be reported. This file can be very large and time consuming to generate, so it is not enabled by default. 

\begin{center}
\begin{tabular}{llp{3.5in}}
\hline
Flag&Long flag&Description\\ \hline
-o  \textit{filename}&{-}{-}output \textit{filename}&Output coverage data base-by-base to a TSV file.\\
&{-}{-}omituncoveredbases&Omit bases from the output file that are not covered by any read\\
-V&{-}{-}verifymapping&Verify that mapping was performed correctly (see below)\\
-S&{-}{-}strict&Use strict verification- only use first coordinate\\
-B \textit{size}&{-}{-}binsize \textit{size}&Bin output data into bins of \textit{size} bases\\
\end{tabular}
\end{center}

Coverage can also be used to verify that mappings have been performed correctly. To do this, reads must be named with a name that starts with the chromosome name and contains the position of the read start and end, each separated with underscores ('\_'). For example, "chr1\_2345\_2388" is valid, as is "chr99\_3350203\_3350299\_long\_name\_9993352:2334". A read is marked as correctly mapped if the mapped chromosome matches the chromosome in the name, and the position is within 5 bases of either of the positions identified in the name (without {-}{-}strict), or within five bases of the first position (with {-}{-}strict). The number of correctly mapped bases is included in the output file, if enabled (ie, if -o and -V are used).

To only report the coverage statistics for each chromosome:
\cmd{openge coverage a.bam}

Or, to output the coverage statistics for each base, and verify the mapping correctness:
\cmd{openge coverage b.bam -o bases.tsv -V}

\subsection {dedup}
The dedup command either marks duplicate reads (compatible with Picard) or removes the duplicate reads. 

Duplicates are marked by setting the 'duplicate' bitflag for each read found to be a duplicate. If the -r or --remove flags are specified, the read is omitted from the output bamfile. The duplicate identification 

\begin{center}
\begin{tabular}{llp{3.5in}}
\hline
Flag&Long flag&Description\\ \hline
-R&{-}{-}removeduplicates&Remove duplicates instead of only marking them.\\
\end{tabular}
\end{center}

For example,
\cmd{openge dedup a.bam -o z.bam} 
identifies duplicates in a.bam, and writes all reads to z.bam with duplicates identified.

\subsection {execute}
\label{execute}

The execute function runs commands in a workflow script. For details of the capabilities of this function and syntax of the script, see the section \ref{pipeline_functionality}.

\begin{center}
\begin{tabular}{llp{4.5in}}
\hline
Flag&Long flag&Description\\ \hline
-t&{-}{-}test&Read and check a workflow without actually executing any commands.\\
&{-}{-}print&Report commands that will be used by the workflow.\\
-p&{-}{-}define&Define a variable to be used in the script.\\
\end{tabular}
\end{center}

When a workflow is excuted with the {-}{-}test flag, no commands will run. This is useful for checking the syntax of a workflow.

When a workflow is executed with the {-}{-}print flag, no commands will run. Instead, a list of commands is listed, as they would have been run. All variables are resolved in the commands. This output may be put in a script, for example, to execute the same commands (but without the features of the workflow manager).

The {-}{-}define parameter may be used to define variables that may be used as part of a workflow. Variables are referenced with \$VARIABLENAME syntax; see the "Workflow functionality" section for more details.

\subsection {help}
The help command shows available options for any OpenGE command. 

\cmd{openge help \textit{[command]}}

For example:

\cmd{openge help mergesort}

\cmd{openge help count}

Available commands can be seen in the table at the top of this section.

\subsection {history}
The history command prints out a list of commands from @PG records in a SAM or BAM file. This can be useful to see how a file was constructed.

There are no command line arguments for this command.

\subsection {mergesort}
\label{mergesort}
This command generates a single file from one or more input files. The supplied input files are merged and then sorted by read position or region.

Parameters:
\begin{center}
\begin{tabular}{llp{3.5in}}
\hline
Flag&Long flag&Description\\ \hline
-r \textit{region}&{-}{-}region \textit{region}&Genomic region to use\\
-q \textit{min\_mapq}&{-}{-}mapq \textit{min\_mapq}&Minimum mapping quality for a read to be included in the mergesort.\\
-n \textit{reads}&{-}{-}n \textit{reads}&Number of reads to use per temporary file. Defaults to 500,000.\\
-C&{-}{-}compresstempfiles&Compress temporary files. By default, temporary files are not compressed.\\
-M&{-}{-}markduplicates&Mark duplicates after sorting.\\
-R&{-}{-}removeduplicates&Mark and remove duplicates after sorting.\\
\end{tabular}
\end{center}

The number of reads per temporary file can have a substantial effect on speed when processing large datasets. Increasing the number of reads per file produces larger (and fewer) temporary files at the expense of memory, but reduces the processing time.

\subsection {localrealign}
This command realigns reads around indels in order to minimize the number of mismatching bases. Since localrealignment uses multiple reads to form a consensus when realigning, it may produce more correct mappings than the original mapper.

This function is intended to be compatible with GATK's IndelRealigner tool.
\begin{center}
\begin{tabular}{llp{3.5in}}
\hline
Flag&Long flag&Description\\ \hline
-R \textit{filename}&{-}{-}reference \textit{filename}&Reference genome in FASTA format. Required.\\
-L \textit{filename}&{-}{-}intervals \textit{filename}&Target intervals to realign. See below for format. Required.\\
\end{tabular}
\end{center}

For example, to realign a file a.bam (generating a.realigned.bam), use the following command:
\cmd{openge localrealign --reference dmel.fasta --intervals a.intervals a.bam -o a.realigned.bam}

\subsubsection {Intervals file format}
The intervals file is a text file listing intervals around which we should realign. Lines must be in the following format:
\textit{chr:\#} or \textit{chr:\#-\#} with no whitespace. This matches the format generated by GATK's RealignerTargetCreator.

\subsection{repeatseq}
RepeatSeq determines genotypes for microsatellite repeats in high-throughput sequencing data. If you use use RepeatSeq, please reference the following manuscript:

G. Highnam, C. Franck, A. Martin, C. Stephens, A. Puthige, and D. Mittelman (2012) Accurate human microsatellite genotypes from high-throughput resequencing data using informed error profiles, Nucleic Acids Res, Epub Oct 22.  [PUBMED ID: 23090981]

Running repeatseq requires an input file, and at least two parameters- a FASTA reference file, and a list of regions.

\begin{center}
\begin{tabular}{llp{3.5in}}
\hline
Flag&Long flag&Description\\ \hline
 -R \textit{filename}&{-}{-}reference \textit{filename}&FASTA reference (required)\\
 -L \textit{filename}&{-}{-}intervals \textit{filename}&Regions file (required)\\
 -l \textit{length}&{-}{-}length \textit{length}&Use only a specific read length or range of read lengths (e.g. LENGTH or MIN:MAX)\\
 -B&{-}{-}before \textit{number}&Required number of reference matching bases BEFORE the repeat (default 3)\\
 -A&{-}{-}after \textit{number}&Required number of reference matching bases AFTER the repeat (default 3)\\
 -Q&{-}{-}quality \textit{number}&Minimum mapping quality for a read to be used for allele determination\\
 -M&{-}{-}multi&Exclude reads flagged with the XT:A:R tag\\
 -P&{-}{-}properlypaired&Exclude reads that are not properly paired (for PE reads only)\\
 -H&{-}{-}haploid&Assume a haploid rather than diploid genome\\
 &{-}{-}repeatseq&Write .repeatseq file containing additional information about reads\\
 &{-}{-}calls&Write .calls file\\
 &{-}{-}tag \textit{name}&Include user-defined tag \textit{name} in the output filename\\
 &{-}{-}flank \textit{number}&Number of flanking bases to output from each read (default 8)\\
 &{-}{-}outfile \textit{filename}&When using stdin as input, set output filename\\
\end{tabular}
\end{center}

When repeatseq is run, it generates a VCF file with the same filename as the input plus a VCF extension; a repeatseq file and calls file may also be generated in the same run.

When piping an input to OpenGE, you must specify the --outfile parameter. This filename is not directly read or written, but used as the base name for the VCF file, and the calls and repeatseq files if generated.

\subsubsection {Output formats}
RepeatSeq can output a VCF file or two custom output formats: .REPEATSEQ and .CALLS. The VCF file is the only file produced by default, however the other two can be enabled through the “{-}{-}repeatseq” and “{-}{-}calls” command line options.

The .calls format is tab delimited and contains repeat annotations, genotypes and confidence values. The genotype field is “NA” when a genotype is not assigned, “N” if the genotype is homozygous with length N, or “NhM” if the genotype is heterozygous with lengths N and M. Similar to the QUAL field in the VCF format, the confidence field here is the phred-scaled quality score for the assertion made in the genotype field, i.e. it gives the value of $-10log_{10} ( P(incorrect-call ))$; high values indicate high confidence calls. The TRF string is the concatenated output of Gary Benson's TRF.

Here is a sample .calls file:

\begin{verbatim}
[region] [TRF string]			        [Genotype][Confidence]
2L:6146-6162    5_3.8_4_78_21_20_52_0_0_47_1.00_ATTA    17        39.3627
2L:7006-7017    3_4.0_3_100_0_24_66_0_0_33_0.92_AAT     NA        NA
2L:10589-10595  1_7.0_1_100_0_14_0_0_0_100_0.00_T       7h6       17.5857
\end{verbatim}

The .repeatseq file contains a full annotated alignment of reads to the a reference repeat. Records consist of the following items:

\begin{enumerate}
\item The first line of each record is the headerline, marked with a \textasciitilde character. The field label keys are as follows:
\begin{center}
\begin{tabular}{lp{5.5in}}
REF&Reference length\\
A&Alleles present\\
C&Concordance (number of reads that support majority allele - 1 / total reads - 1)\\
D&Total number of reads that span the midpoint of the microsatellite (pre-filtering)\\
R&Reads present after all filters\\
S&Number of reads with no CIGAR sequence present (not included in genotyping)\\
M&Average mapping quality of reads used in genotyping\\
GT&Genotype\\
L&Likelihood; similar to VCF format, it gives the value of $-10log_{10}( P(incorrect-call))$\\
\end {tabular}
\end{center}

\item The second line is the reference sequence for quick comparison with read sequences.

\item Each following line is a read that was used in genotyping. It’s space-delimited with the following fields:
\begin{center}
\begin{tabular}{lp{5.5in}}
&Bases leading into the repeat (lowercase 'x' means no base at that position)\\
&Bases of the repeat \\
&Bases following the repeat (lowercase 'x' means no base at that position)\\
&Start position of the read\\
&Read length\\
&Number of consecutive bases that match preceding the repeat\\
&Number of consecutive bases that match following the repeat\\
&Average base quality of the bases of the read (phred scores are converted to base calling probabilities before averaging).\\
M:&Mapping quality of the read\\
F:&Read information flags (p=paired, P=properly paired, u=unmapped, U=mate unmapped, r=reverse strand, R=mate is reversed strand, 1=first mate, 2=second mate, f=failed quality check)\\
C:&CIGAR string\\
ID:&Read-ID\\
\end{tabular}
\end{center}
\end {enumerate}

Here is an example of a .repeatseq file:
\begin{verbatim}
~2L:21828-21836 3_3.0_3_100_0_18_0_0_33_66_0.92_TTG REF:9 A:9 C:1 D:4 R:3 S:0 M:75 GT:9 L:75.84
TGTAAAAT TTGTTGTTG CATCAAAC
TGTAAAAT TTGTTGTTG CATCAAAC 21770 75 8 8 B:0.9949 M:150 F:pPr1 C:75M ID:USI-EAS034:7:8:1591:864#0
TGTAAAAT TTGTTGTTG CATCAAAC 21813 75 8 8 B:0.9957 M:0 F:pPR1 C:75M ID:USI-EAS034:7:5:1114:284#0
xxTAAAAT TTGTTGTTG CATCAAAC 21822 75 6 8 B:0.9957 M:150 F:pPR1 C:75M ID:USI-EAS034:7:79:1478:1338#0

~3R:13465-13477 1_13.0_1_100_0_26_0_0_0_100_0.00_T REF:13 A:10 C:1 D:13 R:8 S:0 M:105.88 GT:10 L:79.89
CCGTTACC TTTTTTTTTTTTT ATCGTTAC
CCGTTACC ---TTTTTTTTTT ATCGTTAC 13413 75 8 8 B:0.9951 M:160 F:pUr1 C:52M3D23M ID:USI-EAS034:7:69:9:117#0
CCGTTACC ---TTTTTTTTTT ATCGTTAC 13431 75 8 8 B:0.9958 M:149 F:pPr2 C:34M3D41M ID:USI-EAS034:7:8:241#0
CCGTTACC ---TTTTTTTTTT ATCGTTAC 13434 75 8 8 B:0.9955 M:151 F:pPR2 C:31M3D44M ID:USI-EAS034:7:17:517#0
CCGTTACC ---TTTTTTTTTT ATCGTTAC 13440 75 8 8 B:0.9956 M:115 F:pPR1 C:25M3D50M ID:USI-EAS034:7:86:1708#0
CCGTTACC ---TTTTTTTTTT ATCGTTAC 13442 75 8 8 B:0.9844 M:103 F:pPR2 C:23M3D51M1S ID:USI-EAS034:7:19:557#0
CCGTTACC ---TTTTTTTTTT ATCGTTAC 13448 75 8 8 B:0.9921 M:80 F:pPR2 C:17M3D58M ID:USI-EAS034:7:60:1435#0
CCGTTACC ---TTTTTTTTTT ATCGTTAC 13449 75 8 8 B:0.996 M:160 F:pPr2 C:16M3D59M ID:USI-EAS034:7:45:467#0
CCGTTACC ---TTTTTTTTTT ATCGTTAC 13456 75 8 8 B:0.9959 M:35 F:pPR2 C:9M3D66M ID:USI-EAS034:7:99:1447#0

~X:318-324 3_2.3_3_100_0_14_0_57_42_0_0.99_GCC REF:7 A:7[3] 10[2] C:0.5 D:9 R:5 S:0 M:54 GT:7h10 L:36.04
GTGGTGGT G---CCGCCG TTGATTTG
GTGGTGGT G---CCGCCG TTGATTTG 3149184 75 8 8 B:0.9949 M:160 F:pPr1 C:15M3I57M ID:USI-EAS034:7:56:544#0
GTGGTGGT GGTGCCGCCG TTGATTTG 3149189 75 8 8 B:0.812 M:0 F:pPr1 C:22S30M3I20M ID:USI-EAS034:7:30:128#0
GTGGTGGT GGTGCCGCCG TTGATTTx 3149192 75 8 7 B:0.7287 M:0 F:pPr2 C:32S27M3I13M ID:USI-EAS034:7:83:1423#0
GTGGTGGT G---CCGCCG TTGATTTG 3149195 75 8 8 B:0.8156 M:13 F:pPr1 C:21S4M3I47M ID:USI-EAS034:7:35:457#0
xxGGTGGT G---CCGCCG TTGATTTG 3149212 75 6 8 B:0.8486 M:150 F:pPR1 C:58M17S ID:USI-EAS034:7:55:1592#0
\end{verbatim}

\subsection{stats}
Prints out statistics about a file. No flags are required.
The stats command generates statistics for one or more input files. The computed statistics are summarized below.

Example:

\cmd{openge stats a.bam}
\cmd{openge stats a.bam b.bam c.bam {-}{-}inserts {-}{-}lengths {-}{-}verbose}
Parameters:
\begin{center}
\begin{tabular}{llp{3.5in}}
\hline
Flag&Long flag&Description\\ \hline
-I&{-}{-}inserts&Include detailed stats about inserts.\\
-L&{-}{-}lengths&Include the distribution of read sizes.\\
\end{tabular}
\end{center}

For example:
\cmd {openge stats a.bam b.bam -I -L}
gives results similar to this:
\cmd {Total reads:            15419\\
Mapped reads:           14812 ( 96.1\%)\\
Forward strand:          7901 ( 51.2\%)\\
Reverse strand:          7518 ( 48.8\%)\\
Failed QC:                  0 (  0.0\%)\\
Duplicates:                 0 (  0.0\%)\\
Paired-end reads:       15419 (100.0\%)\\
'Proper-pairs':         13856 ( 89.9\%)\\
Both pairs mapped:      14205 ( 92.1\%)\\
Read 1:                  7697\\
Read 2:                  7722\\
Singletons:               607 (  3.9\%)\\
Read lengths:\\
    45bp:                2327 ( 15.1\%)\\
    64bp:                1699 ( 11.0\%)\\
    75bp:               11393 ( 73.9\%)\\
Insert size (absolute value):
    Mean:               218.7
    Median:             189.0}

\subsection{version}
Prints out the version number and the date and time that this version was compiled.

\subsection{view}
The view command returns a portion of a file optionally restricted to:
\begin{itemize}
\item Reads mapped to a specified chromosome, or region of a chromosome ({-}{-}region)
\item Reads with Mapq values above a threshold ({-}{-}mapq)
\item Reads of a certain length or range of lengths ({-}{-}length)
\item Limited to a certain number of results ({-}{-}count)
\end{itemize}
or any combination of the above.

The view command can also trim bases from the beginning or end of reads. If the amount being trimmed from the beginning and end combined exceeds the length of the read, the read will be omitted.

Parameters:
\begin{center}
\begin{tabular}{llp{3.5in}}
\hline
Flag&Long flag&Description\\ \hline
-n \textit{number}&{-}{-}count\textit{number}&Number of reads to include in the generated file. Defaults to include the entire file.\\
-r \textit{region}&{-}{-}region \textit{region}&Region string (see below for format)\\
-q \textit{min\_mapq}&{-}{-}mapq \textit{min\_mapq}&Minimum mapq allowed\\
-l \textit{length}&{-}{-}length \textit{length}&Allowed range of read lengths (see below for format)\\
-B \textit{length}&{-}{-}trimbegin \textit{length}&Trim \textit{length} bases from the beginning of all reads\\
-E \textit{length}&{-}{-}trimend \textit{length}&Trim \textit{length} bases from the end of all reads\\
\end{tabular}
\end{center}

Trimming with the {-}{-}trimbegin and {-}{-}trimend parameters is only supported for the FASTQ output format at this time.

\subsubsection{Region string format}
Region strings are formatted similarly to the equivalent bamtools region strings, and this section is an excerpt from the bamtools documentation.

A proper region string can be formatted like any of the following examples:

\begin{center}
\begin{tabular}{lp{3.5in}}
{-}{-}region chr1&only alignments on (entire) reference 'chr1'\\
{-}{-}region chr1:500&only alignments overlapping the region starting at chr1:500 and continuing to the end of chr1\\
{-}{-}region chr1:500..1000&only alignments overlapping the region starting at chr1:500 and continuing to chr1:1000\\
\end{tabular}
\end{center}

\subsubsection{Length format strings}
The following formats can be used to 
\begin{center}
\begin{tabular}{lp{3.5in}}
{-}{-}length 64&Only allow reads with length of 64\\
{-}{-}length +64&Only allow reads with length greater than 64\\
{-}{-}length -64&Only allow reads with length less than 64\\
{-}{-}length 64-72&Only allow reads with length between 64 and 72 (inclusive)\\
\end{tabular}
\end{center}

\section {Performance tuning}

\subsection {mergesort}
\label{mergesort_tuning}
When sorting larger files, mergesort stores intermediate results in a number of temporary files. The performance of reading and writing these files is the biggest contributor to mergesort's performance, and a few settings are available can tune mergesort to run better on your system.

\begin{itemize}
\item The \textbf{{-}{-}compresstempfiles} (section \ref{mergesort})  controls whether or not temporary files are compressed. Selecting this option reduces the space consumed by temporary files down by approximately 60\%, at the cost of extra CPU time. If performance is limited by disk speed, try this option.
\item The \textbf{{-}{-}tempdir} (section \ref{general_options})  option allows you to specify a directory to store temporary files in. By default, /tmp is used. If you have a faster disk installed for temporary files (especially a SSD), putting the temporary files on that disk may increase performance.
\item \textbf{{-}{-}n} (section \ref{mergesort}) allows you to specify a number of reads to put in each tempfile. This parameter affects both the number of temporary files, and the amount of RAM consumed while running.

Typical values for this parameter will be about 500,000 for a laptop/desktop system, up to 5,000,000 or larger for a high-RAM system (50GB+).

If this number is set to low, the number of temporary files will exceed the maximum number of files that you can have open, and a warning to this effect will be displayed. On linux, this limit can be found and increased (a bit) using: \cmd{ulimit -n 2048} See this page for more information:
\cmd{\href{http://stackoverflow.com/questions/34588/how-do-i-change-the-number-of-open-files-limit-in-linux}{http://stackoverflow.com/questions/34588/how-do-i-change-the-number-of-open-files-limit-in-linux}}

If the number is too high, openge will use all of your system's RAM.
\end{itemize}

\section {Workflow functionality}
\subsection {Overview of functionality}
\label{pipeline_functionality}
OpenGE also provides workflow manager, under the "execute" command (see the previous section). This tool is capable of running a sequence of commands, checking for successful execution, and managing naming of intermediate files automatically.

Syntax and functionality is based on the open source Bpipe tool (\url{http://code.google.com/p/bpipe}), with which we attempt to maximize compatibility.

\subsection {Getting started}
Workflow scripts are run as files passed to OpenGE's execute command:

\cmd{openge execute \textit{workflow input\_file}}

See the execute documentation (section \ref{execute}) for more details on how to call the execute command.

We can best demonstrate some workflow features and syntax with examples. For some features, the Bpipe wiki (\url{http://code.google.com/p/bpipe/wiki/GettingStarted}) provides more details.

\subsection {Simple example: stages, \$input and \$output}
A workflow is a series of stages, each with a single purpose. Outputs from earlier stages are passed to later stages as inputs in a chain; if any stage is not successful, the stages after it won't be executed and an error message will be printed on the command line.

A stage is defined with a simple name, and an exec line:
\begin{verbatim}
index = {
  exec "samtools index input"
}
\end{verbatim}

The \$input variable will be automatically filled in during execution, as well as \$output. Output filenames are automatically generated from the \$input filename and the stage name. This output filename is both used in the current stage as \$output, as well as the \$input for the next stage.

If you don't specify otherwise (see the next section), the $output will simply be the $input with name of the stage appended.

Stages are combined together in a run block in order to create a pipeline. To execute stages sequentially, join them with a plus sign:
\begin{verbatim}
run {
  sort + index
}
\end{verbatim}
Run blocks allow you to easily configure your pipeline by adding and removing stages without having to worry about updating filenames. Putting the above together, we could come up with a script that will sort an input BAM or SAM file, and then index the BAM that mergesort generates:

\begin{verbatim}
sort = {
  exec "./openge mergesort $input -o $output -v"
}
index = {
  exec "samtools index $input"
}
run {
  sort + index
}
\end{verbatim}

You can run this example file like this:

\cmd{openge execute example.pipeline in.bam}

Samtools must be installed for the pipeline to function correctly, and you should replace in.bam with the name of the BAM file that you want to sort.

\subsection {A more complex example: variables and filter()}

\subsubsection{Message statements and forward input}
In the previous section, we showed how you can use exec statements to run external commands. In reality there are three things that can appear inside a workflow stage:
\begin{itemize}
\item exec statments
\item msg statements. Message statements are used to print a message on the console while the workflow is running. This might be a status message, or a reminder of what parameters are being used.  They use similar syntax to execute statements:
\cmd{msg "Hello world!"}
\item The forward input statement. Stages normally forward on the output filename of the stage (\$output) as  the input to the next stage. Including the forward input statement passes along the \$input instead.
\cmd{forward input}
This is useful in situations where the output of the stage shouldn't be used explicitly later in the workflow. Indexing a BAM file is a great example of this.
\end{itemize}

\subsubsection{Variables}
Variables can be defined to add parameters to your workflow. They can either be defined at the top of your workflow file like:
\cmd{LRREF = "dmel.fa"}
or can be specified on the command line.
\cmd{openge execute a.workflow -p NAME=Lee}

These variables can appear in any exec or msg statement.

\subsubsection{produce(), filter(), and transform()}
Often, the default \$output name isn't exactly what you want. Produce(), filter(), and transform() are used to control the output filename of a stage. Each requires a single quoted string. For example:

\begin{verbatim}
sort = {
  filter("sort") {
    exec "./openge mergesort $input -o $output -v"
  }
}
\end{verbatim}

There are three easy ways to change how \$output is generated:
\begin{itemize}
\item The output filename can be directly specified with the produce statement. A block with
\cmd{produce("file.bam") \{ ... \} }
will always have \$output set as "file.bam".
\item Filter() can be used to generate a filename with an additional string added, preserving the original extension. For example, with an \$input of "some\_file.bam",
\cmd{filter("sorted") \{ ... \} }
will produce a file called "some\_file.sorted.bam". This is best suited to stages that filter a file without changing its type.
\item Transform() can be used to set the \$output to match the \$input, but with a different extension. For example, with an \$input of "some\_file.sorted.bam",
\cmd{transform("vcf") \{ ... \} }
will produce a file called "some\_file.sorted.vcf". This is most useful for stages that produce one file type from another, either by converting, the file, or by generating some derived data from the input file.
\end{itemize}
\subsubsection{Comments}
Execute also uses C++-style comments. Comments can be added either between /* and */ symbols, or between a // symbol and the end of the line.
\cmd{// this is a comment that lasts until the end of the line}
\cmd{/* this is a comment that lasts until\newline this symbol: */}

\subsubsection{Example source code}
If we bring together all of the above concepts, we can construct useful workflow for analying genome data:
\begin{verbatim}
//at the top we define reference files that are used in the below stages...
LRREF = "dmel.fa"
LRINTERVALS = "801.bwa.merged.bam.sorted.bam.dedup.intervals"
RSREGIONS = "fly.regions"

/* now on to the actual code */
sort = {
  filter("sort") {
    msg "Sorting file."
    exec "./openge mergesort $input -o $output -v"
  }
}
dedup = {
  filter("dedup") {
    msg "Marking duplicates."
    exec "./openge dedup $input -o $output -v"
  }
}
lr = {
  filter("lr") {
    msg "Realigning reads near indels."
    exec "./openge localrealign --reference $LRREF -v --intervals $LRINTERVALS $input -o $output"
  }
}
index = {
  exec "samtools index $input"
  forward input
}
repeatseq = {
  transform("repeatseq") {
    exec "./openge repeatseq $input -R $LRREF -L $RSREGIONS --repeatseq --calls"
  }
}
run {
  sort + index + dedup + index + lr + index + repeatseq
}
\end{verbatim}

\subsection {Bpipe compatibility}
Our goal is to provide a high level of compatibility with the Bpipe tool and with existing workflow scripts. In later versions of OpenGE, we plan on introducing additional features that Bpipe provides. Many new features are planned for future releases of OpenGE.

\subsubsection {Compatibility matrix}
Some notable Bpipe features that we do not plan to bring OpenGE are listed below, along with suggestions for how to replace them:
\begin{center}
\begin{tabular}{p{1.9in}p{3.5in}}
\hline
Type&Replacement\\ \hline
Annotation-style statements:
\cmd{@Filter()}
\cmd{@From()}
\cmd{@Produce()}
\cmd{@Transform()}&
No annotation-style statments are accepted. These statements can be replaced with function-style replacements.\\ \\

External stages via load()\newline External stages from \$BPIPELIB folders\newline External stages from \textasciitilde/bpipes&Loading external scripts is not supported in this version, but is planned for later versions. If you need this functionality, we recommend you use Bpipe instead.\\ \\

Inline Java code\newline Inline Groovy code&
Inline Java and Groovy code is not supported. If you need this functionality, we recommend you use Bpipe instead.\\ \\

XMPP notifications\newline Email notifications&Notifications are not supported. If you need this functionality, we recommend you use Bpipe instead.\\

\end{tabular}
\end{center}

\section {License}
This software is made available through the Virginia Tech non-commerical license. See openge/license.txt for more details. 

\copyright 2012 Virginia Bioinformatics Institute. All rights reserved.
\end{document}
