\documentclass[11pt]{article}

\usepackage[parfill]{parskip}
\usepackage{comment}
\usepackage{hyperref}

\setlength{\topmargin}{-.5in}
\setlength{\textheight}{9in}
\setlength{\oddsidemargin}{.125in}
\setlength{\textwidth}{6.25in}

\newcommand {\cmd}[1] {\begin{quote}#1\end{quote}}

\begin{document}
\title{Open Genomics Engine - Developer Documentation}
\author{Lee Baker and David Mittelman\\
Virginia Bioinformatics Institute \\
leecb@vt.edu, david.mittelman@vt.edu}
\maketitle

\section {Compiling OpenGE}
Building OpenGE is tested on Ubuntu Linux and Mac OS X, and should be compatible with most other POSIX systems.

\subsection{Ubuntu}
These steps have been tested on Ubuntu 11.10, and are likely similar for other versions.

Ensure that you have the following packages installed: cmake, libboost-program-options-dev, and have these tools installed: gcc, make, git. Gcc and make are part of the build-essentials package, which you should probably install if you haven't already.

Check out the source code from github:
\cmd{git clone git@github.com:adaptivegenome/openge.git}

Open the directory containing the code you just checked out- it should be called openge. Make a folder inside the directory called 'build'. This is where cmake will put temporary build files. Generate those build files using cmake.
\cmd{cd openge\\mkdir build\\cd build\\cmake ..}

Compile the code.
\cmd{make}
The resulting binary will be in the openge/bin/ directory.

\subsection{Mac}
These steps are tested on Mac OS X 10.7, and are likely similar for other versions.

Ensure that you have a C and C++ compiler installed; the easiest way to do this is to install Apple's XCode (free from the App Store), and then download the command line tools (found inside XCode on the XCode menu - Preferences - Downloads - Components - Command line tools). If you choose to use XCode's tools or the XCode IDE, see the section below on XCode and Mac.

Install CMake if you don't already have it. Precompiled binaries can be found at \url{http://www.cmake.org/cmake/resources/software.html}.

Download, compile and install the boost libraries from \url{http://boost.org}.

Install git if you don't have it. \url{http://git-scm.com/}

Check out the source code from github:
\cmd{git clone git@github.com:adaptivegenome/openge.git}

Open the directory containing the code you just checked out- it should be called openge. Make a folder inside the directory called 'build'. This is where cmake will put temporary build files. Generate those build files using cmake.
\cmd{cd openge\\mkdir build\\cd build\\cmake ..}

Compile the code.
\cmd{make}
The resulting binary will be in the openge/bin/ directory.

\subsubsection{Using XCode on Mac}
Apple's XCode can be used to compile OpenGE. To use XCode, perform the same steps as above, up until cmake. Use the following flags instead:
\cmd{cmake -G Xcode ..}
and a project file called OpenGE.xcodeproj will be created. Open this in Xcode.

Note: there are issues when using XCode 4.3 or later with out of date versions of cmake- generally cmake cannot find a compiler, or has issues with the c compiler it does find. To fix this, update to CMake 2.8.8 or later.

\subsection{Dependencies}
The following software packages should be on your machine before attempting to compile OpenGE:
\begin{itemize}
\item standard build tools: gcc, g++, make
\item Cmake
\item boost::program\_options. On Mac, you can compile this from source using the commands  (see \url{http://www.boost.org/doc/libs/1_49_0/doc/html/program_options.html})
\end{itemize}

\section {Contributing OpenGE}
Contributions of code or documentation are welcome. Contact us to learn more about how you can get involved at evolvability@vt.edu or interact with us through our github page.

\subsection {OpenGE on Github}
You can find OpenGE on Github at \url{https://github.com/adaptivegenome/openge}. This is where you should post issues that you have with OpenGE, and submit a pull request for any code that you may want to contribute.
\subsection {Adding a command}
Commands are represented inside OpenGE as a subclass of OpenGECommand. You should do the following things:
\begin{itemize}
\item Create a class definition for your command inside openge/src/commands/commands.h, similar to the definitions already in place.
\item Create a cpp file in the openge/src/commands/ directory that contains the implementation of your subclass. Your implementation will have two functions included- 
\cmd {void CountCommand::getOptions()}
is where you should provide definitions of your command line options, and 
\cmd{int CountCommand::runCommand()}
which is called after all command line options are parsed, and should contain the code to actually run your command. Look at the 'stats' command in command\_stats.cpp for a good example of what this class should look like.
\item Create a reference to your class in CommandMarshall::commandWithName in openge/src/commands/commands.cpp so that your command can be called from the command line.
\end{itemize}

Most commands are constructed of several modules, such as FileReader, Count, ReadSorter, that perform some action on a stream of commands. This allows us to write a new command quickly, reusing much of the common code for common tasks. Modules are in the openge/src/modules directory, and are all subclasses of AlgorithmModule.

If you are adding an algorithm to OpenGE, it is strongly recommended that you implement your algorithm as a subclass of AlgorithmModule. There are a handful of modules included with OpenGE that you can look at as examples.

\section {Get involved}

OpenGE discussion forum: \url{http://seqanswers.com/forums/forumdisplay.php?f=43}

Project website: \url{http://opengenomicsengine.org/}

Lab website: \url{http://genome.vbi.vt.edu/}

OpenGE authors:
\cmd{David Mittelman, VBI: david.mittelman@vt.edu\\Lee Baker, VBI: leecb@vt.edu}

\end{document}